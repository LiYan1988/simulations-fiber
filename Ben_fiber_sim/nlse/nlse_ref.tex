\documentclass[a4paper, 11pt]{article}
%\documentclass[a4paper, 11pt, draft]{article}
\usepackage[latin1]{inputenc}
\usepackage{a4wide}
\usepackage{graphicx}
\usepackage[hang,bf]{caption}
\usepackage{amsmath}
\usepackage{srcltx}
\usepackage{verbatim}

\newcommand{\odn}[3]{\frac{d^{#3}#1}{d#2^{#3}}}
\newcommand{\od}[2]{\frac{d#1}{d#2}}
\newcommand{\pdn}[3]{\frac{\partial^{#3}#1}{\partial#2^{#3}}}
\newcommand{\pd}[2]{\frac{\partial#1}{\partial#2}}
\newcommand{\sech}{\operatorname{sech}}
\newcommand{\sign}{\operatorname{sign}}
\newcommand{\Ai}{\operatorname{Ai}}
\newcommand{\Bi}{\operatorname{Bi}}
\newcommand{\F}{\mathcal{F}}
\newcommand{\Finv}{\mathcal{F}^{-1}}
\newcommand{\idxtxt}[1]{{\textrm{\scriptsize{#1}}}}
\newcommand{\fourier}{\mathcal{F}}
\newcommand{\invfourier}{\mathcal{F}^{-1}}
\setlength\arraycolsep{1.3pt}

\title{Documentation for the NLSE solvers}
\author{Pontus Johannisson}

\begin{document}
\maketitle

\section{NLSE, SPM, 1D}
In the case with a single polarization, the NLSE is
\begin{equation}
i \pd{u}{z} = \frac{\beta_2}{2} \pdn{u}{t}{2} - i \frac{\alpha}{2} u -
\gamma |u|^2 u .
\end{equation}
The notation is taken from Agrawal's ``Nonlinear fiber optics''. By
discarding the nonlinear term and Fourier transforming we obtain
\begin{equation}
i \od{U}{z} = -\frac{\beta_2 \omega^2}{2} U - i \frac{\alpha}{2} U,
\end{equation}
or
\begin{equation}
\od{U}{z} = \frac{i \beta_2 \omega^2}{2} U - \frac{\alpha}{2} U =
\frac{1}{2} ( i \beta_2 \omega^2 - \alpha ) U,
\end{equation}
with the solution
\begin{equation}
U = U_0 \exp \left[ \frac{1}{2} ( i \beta_2 \omega^2 - \alpha ) z
\right].
\end{equation}
Thus, the dispersive step of length $dz$ is made according to
\begin{equation}
u(z + dz) = \invfourier \left[ \fourier[u(z)] \exp \left( \frac{1}{2}
  ( i \beta_2 \omega^2 - \alpha ) dz \right) \right].
\end{equation}
By including only the nonlinear term we have
\begin{equation}
i \od{u}{z} = - \gamma |u|^2 u,
\end{equation}
with the solution
\begin{equation}
u = u_0 \exp (i \gamma |u_0|^2 z).
\end{equation}
Thus
\begin{equation}
u(z + dz) = u(z) \exp (i \gamma |u(z)^2| \, dz).
\end{equation}
This procedure is implemented in {\texttt{nls\_spm.m}}.

\verbatiminput{nls_solv.m}

\subsection{Test case---the fundamental soliton}
If attenuation is neglected, the fundamental soliton solution is
\begin{equation}
u = \sqrt{-\frac{\beta_2}{\gamma t_0^2}} \sech \left( \frac{t}{t_0}
\right) \exp \left( -i \frac{\beta_2}{2 t_0^2} z \right),
\end{equation}
or
\begin{equation}
u = A_0 \sech \left( A_0 \sqrt{-\frac{\gamma}{\beta_2}} \, t \right)
\exp \left( i \frac{\gamma A_0^2}{2} z \right).
\end{equation}
(In the first case we have chosen the pulse width. In the second case
we have chosen the amplitude.) This is implemented in
{\texttt{fundamental\_soliton.m}}.

\section{The Manakov model}
The Manakov equation is
\begin{equation}
i \pd{u}{z} = \frac{\beta_2}{2} \pdn{u}{t}{2} - i \frac{\alpha}{2} u -
\gamma (u^\dag u) u,
\end{equation}
where the vector $u = [u_x, u_y]^T$. This equation ``accurately
describes nonlinear and dispersive light propagation in standard
communication fiber with rapidly and randomly varying birefringence
when polarization mode dispersion can be neglected'' [Wang, Menyuk,
1999].  The linear and nonlinear steps consist of two separate
operations of the type described above.
\end{document}
